\section{Memory Consistency Model}

%TODO (Riyadh): modify the PENCIL spec to say that it requires a C11 compiler
% instead of a C99 compiler.

The \pencil memory model tells programmers what they can expect from an OpenCL
implementation; which memory operations are guaranteed to happen in which order
and which memory values each read operation will return.
The memory model also tells compiler writers which restrictions they must follow
when implementing compiler optimizations.

The memory consistency model in \pencil is inspired from the memory model of
the ISO C11~\ref{c11} and the OpenCL 2.0~\ref{opencl2} programming languages.
The main difference between the \pencil memory model and the C11/OpenCL 2.0
memory models is that \pencil has sequential semantics whereas C11 and OpenCL
2.0 do not.

Let us assume that R(a) means a read to the memory location a and W(a) means
a write to a location a.
In a given program, for any two statements that access two different memory
locations a and b, there are four possible reorderings:
- Reordering an R(a) followed by an R(b);
- Reordering an R(a) followed by a  W(b);
- Reordering a  W(a) followed by an R(b);
- Reordering a  W(a) followed by a  W(b);

In \pencil, two consistency memory models exist:
%TODO (ulysse): provide a correct definition.
\begin{itemize}
  \item Relaxed consistency: assuming that two statements that access two
  different memory locations a and b are marked to have relaxed consistency,
  then the compiler is allowed do any reordering between these
  two statements.
  \item Sequential consistency: assuming that two statements that access two
  different memory locations a and b are marked to have \eph{sequential
  consistency}, if the code generated from \pencil is a parallel code
  (i.e. the code is split to be executed by multiple threads), then:
 	- if the two statements are executed by the same thread,
 	no reordering of the two statements is possible.
  	- if the two statements are executed by different threads, then this
  	execution is required to appear as if one of the statements is executed
  	first and then the second statement is executed (interleaving
  	of execution).
\end{itemize}

\pencil defines the enumerated type \lstlisting!memory_order!
which defines these two memory models:
\lstinline!memory_order_relaxed! stands for the relaxed consistency memory
model and \lstinline!memory_order_seq_cst! stands for the sequential
consistency memory model.

\lstinline!enum memory_order {
    memory_order_relaxed,
    memory_order_seq_cst
};!

\subsection{Atomic Operations and Memory Fences}

There are two types of synchronization operations: \emph{atomic operations} and
\emph{fences}.

\subsubsection{Atomic Operations}
Atomic operations are indivisible. They either occure completely or not at all.
These operations are used to order memory operations between units of execution
and hence they are parameterized by the \pencil memory consistency model
(they can follow either the relaxed consistency model or the seuential
consistency model).
The atomic operations for \pencil are similar to the corresponding operations
defined by the C11 standard~\ref{c11}.
The \pencil atomic operations apply to variables of an atomic type.

The following C11 atomic builtins are defined in \pencil:
atomic_init, atomic_store, atomic_store_explicit, atomic_load,
atomic_load_explicit, atomic_exchange, 	atomic_exchange_explicit,
atomic_compare_exchange_strong, atomic_compare_exchange_strong_explicit,
atomic_compare_exchange_weak, atomic_compare_exchange_weak_explicit,
atomic_fetch_key, atomic_fetch_key_explicit (where key can be either
add, sub, or, xor, and, min, max),
atomic_flag_test_and_set, atomic_flag_test_and_set_explicit,
atomic_flag_clear, atomic_flag_clear_explicit.

They have exactly the same semantics as their equivalent C11 functions.
The other C11 atomic functions are not defined in \pencil as they do not have
an equivalent in OpenCL.

\subsubsection{Memory Fences}
A memory fence is an atomic operation that is not associated to a memory
location.

\lstinline!void pencil_fence();!

%TODO (ulysse): Explain the semantics of a fence.
%TODO (ulysse): Provide an example.

\subsection{Atomic Types and Initialization}
\pencil supports the following atomic types defined by C11 (since these are
also supported by OpenCL 2.0):
atomic_flag, atomic_int, atomic_uint, atomic_long, atomic_ulong, atomic_llong,
atomic_ullong, atomic_char16_t, atomic_char32_t, atomic_wchar_t and
atomic_size_t.

The C11 ATOMIC_VAR_INIT and ATOMIC_FLAG_INIT macros are also both defined
in \pencil.

%TODO (ulysse): envoyer ce fichier sur pencil-discussion [AT] inria.fr pour que
% les autres donnent leurs avis et faire des petites modifications s'ils
% proposent des modifications.

