\section{Introduction}

Many systems --- from supercomputer installations to embedded
systems-on-chip --- benefit from using special-purpose
{\em accelerators} which can significantly outperform general-purpose
processors in terms of energy efficiency as well as
execution speed.
Software for such systems, however, is currently written using
low-level APIs\nomenclature{APIs}{Application Program Interface}
such as OpenCL (Open Computing Language)\nomenclature{OpenCL}{Open Computing Language}~\cite{stoneopencl} and
CUDA (Compute Unified Device Architecture)\nomenclature{CUDA}{Compute Unified Device Architecture}~\cite{cudaref},
which increases the cost of
its development and maintenance.
A compelling alternative for developers is to work with higher-level
programming languages, and to leverage compilation technology to automatically
generate efficient low level code.

For general-purpose languages in the C family, this approach is hindered
by the difficulty of static analysis in the presence of pointer aliasing,
data-dependent array accesses and dynamic control.
For example, the possibility of aliasing often forces a parallelizing compiler
to assume that it is \emph{not} safe to parallelize a region of source
code, even though aliasing might not actually
occur at runtime.

Domain-specific languages (DSLs\nomenclature{DSL}{Domain-Specific Languages})
can circumvent this problem: it is often
clear how parallelism can be exploited given
high-level knowledge about standard operations in a given domain such as
linear algebra~\cite{vobla2014},
image processing~\cite{DBLP:conf/pldi/Ragan-KelleyBAPDA13}
or partial differential equations~\cite{DBLP:journals/toms/AlnaesLORW14}.
The drawback of the DSL approach is the significant effort required to
lower code all the way from the DSL level to highly optimized OpenCL or CUDA.
The effort involved is even more significant if optimization is required
for multiple platforms.
Given typical budget constraints, the DSL implementers will
likely limit their efforts to a set of techniques useful for a small
number of target platforms, thus
compromising on performance portability.  Moreover, the implementers
of different DSLs will likely spend their efforts on implementing an
overlapping set of techniques.
The existence of a common intermediate language serving as a target
for DSL compilers would reduce considerably the development costs.
In addition, if this language is a high level language, it can also
be used directly by domain experts to target accelerators and the
advantage is doubled.

% To some extent, the design of such an intermediate
% language obeys the same motivations that led to the design of compiler
% middle-ends that support a common set of analyses and optimizations
% for a variety of programming languages and target instruction
% sets. However, the technical context is very different, involving
% code generation for complex memory hierarchies and massively parallel
% hardware accelerators. This involves program analyses and
% transformations on multidimensional iteration spaces and arrays,
% dealing with mapping, scheduling, and automatic data transfers in a
% massively concurrent context.
% 
% Beside enhancing productivity, DSLs have the advantage of using high level
% constructs that have rich semantics.  These constructs provide a wealth of
% information that enable the compiler to optimize and parallelize
% the code even for algorithms that are considered to be irregular
% when expressed in languages like C (e.g., operations on lists, like map).
% %
% DSL compilers maintain tight control over the generated code, eliminating
% many of the problems faced by general-purpose optimizing compilers.

We present the design and implementation of \pencil, a plat\-form-neu\-tral 
compute intermediate language.
\pencil aims to serve both as a portable implementation language to facilitate 
the acceleration of new and legacy applications on modern accelerators, and as 
a intermediate language for DSL compilers.


\section{\pencil Language}

\subsection{Overview of \pencil \label{pencil-overview}}

\pencil is a rigorously-defined subset of C99~\cite{c99}%
\nomenclature{C99}{C Standard 1999}.  It enforces a set of
coding rules principally related to restricting the manner in which pointers
can be manipulated.  These restrictions make \pencil code
``static analysis-friendly'': the rules are designed to enable a compiler
to perform better optimization and parallelization when translating \pencil
to a lower-level formalism such as OpenCL.
\pencil is also equipped with specific language constructs, including
\emph{assume predicates} and \emph{side effect summaries} for functions,
that enable communication of domain-specific information and static properties
to the \pencil compiler, to be used for parallelization and optimization.
These specific constructs provide information that is difficult for a
compiler to extract from arbitrary code but that can be easily captured from a DSL, or
expressed by a programmer.
Although the target platforms are highly parallel,
\pencil deliberately has sequential C99 semantics in order to simplify DSL
compiler development and the work of a domain expert directly
developing in \pencil, and more importantly, to avoid committing to
any particular pattern(s) of parallelism.

Where necessary, \pencil exploits the flexibility of non-C99 extensions,
and particularly GNU
C extensions~\cite{gccguide} such as type attributes.
A design goal was to avoid pragma-based directives as directives are still
not considered to be first class citizens by many compilers.
However, in very few cases, \pencil relies on directives to attach
properties to a control flow region of the code, no better C-compatible
alternative being available.
These directives have been inspired by standard directives used
for vectorization and thread-level parallelism, but retain a strictly
sequential semantics in \pencil.

Because it is based on C, the learning curve for \pencil is gentle.
By design, \pencil interfaces with C code, so that legacy C 
applications can be
incrementally ported to \pencil.  From the point of view of DSL
compilation, \pencil offers an easy-to-target intermediate language because all a DSL-to-\pencil
compiler must do is faithfully encode the semantics of the input DSL program
into \pencil; auto-parallelization and optimization for multiple accelerator
targets is then taken care of by the downstream \pencil compiler.
Because DSL-to-\pencil compilers have tight control over the code they
generate, such compilers can aid the effectiveness of the downstream
\pencil compiler by careful generation of code, and by communicating
domain-specific information via the language constructs \pencil provides
for this purpose.


\paragraph*{Design considerations}
The design of \pencil is guided by the following considerations:
\begin{itemize}
\item \pencil should have sequential semantics to facilitate the
  design and implementation of domain-specific compilers targeting
  \pencil, to ease the work of \pencil programmers, and to avoid committing
  early to target-specific patterns of parallelism.


\item \pencil should simplify static code analysis for the optimizing compiler.
  For example, the use of pointers is disallowed, except in specific cases,
  relieving the compiler from issues related to aliasing.

\item \pencil should provide facilities that allow a DSL-to-\pencil
  compiler to convey, in the \pencil code that it generates,
  domain-specific information that can be exploited by the
  compiler to perform better optimizations.  For example, \pencil
  should allow the user to indicate that the size of an array does not
  exceed a specific size to enable the compiler to place
  that array in the shared memory of a GPU (Graphics Processing Unit).

\item For compatibility reasons, a standard C99 compiler that supports
  GNU C attributes~\cite{gccguide} should be able to compile \pencil, this allows for
  greater portability and makes the debugging of \pencil code easier.

\item The subset of C99 that constitutes \pencil should
  be as large as the above design  considerations permit.
  A very small and restrictive subset limits the reuse and
  modularity of \pencil code, and makes
  \pencil less attractive to programmers and
  DSL compiler-writers.

\item Language extensions (compared to C99) should be minimized.
  Too many extensions make it harder for compilers
  to support \pencil.

\item \pencil code should be able to interface with non-\pencil code
  and external library functions.
\end{itemize}

Three possible scenarios of using \pencil are possible:
\begin{enumerate}
\item \pencil code is generated by a DSL compiler;
\item \pencil code is written by a programmer;
\item a mixture of both scenarios.
\end{enumerate}

\begin{figure}[ht]
{
 \centering
\begin{tikzpicture}[auto,node distance=2.7cm,
block/.style={thick,draw,fill=gray!20,text centered,rounded corners,%
align=center,minimum height=4em},
empty/.style={minimum height=4em},
link/.style={->,line width=3pt,>=stealth}]
\matrix[column sep=4mm]{
\node[block](GPU) {GPUs\\(Nvidia, AMD,\\ARM, \ldots)};
&
\node[block](CPU) {CPUs\\(Intel, AMD, \ldots)};
&
\node[block](FPGA) {FPGAs\\(Altera, Xilinx, \ldots)};
&
\node[block](other) {Other\\accelerators};
\\
};
\node (center) [fit=(CPU.east)(FPGA.west)] {};
\node[block,above of=center,text width=11em] (OpenCL) {OpenCL};
\node[block,above of=OpenCL] (pencil) {\pencil \\
Platform-Neutral Compute\\Intermediate Language};
\node[block,above of=pencil] (DSL) {Domain Specific Languages};
\draw[link] (OpenCL.south west) -- (GPU);
\draw[link] (OpenCL) -- (CPU);
\draw[link] (OpenCL) -- (FPGA);
\draw[link] (OpenCL.south east) -- (other);
\draw[link] (pencil) to node[pos=0.5,left,align=center]
{Optimizing, auto-parallelizing\\\pencil $\to$ OpenCL compiler} (OpenCL);
\draw[link] (DSL) to node[pos=0.5] {DSL $\to$ \pencil compilers} (pencil);
\node[left=1cm of OpenCL,align=center,anchor=east] (direct)
{Direct OpenCL\\programming};
\draw[link] (direct) -- (OpenCL);
\node[right=1cm of pencil,align=left,anchor=west] (direct2)
{Direct \pencil\\programming\\(hand written\\\pencil code)};
\draw[link] (direct2) -- (pencil);
\end{tikzpicture}
 \caption{High level overview of our vision for the use of \pencil}
 \label{fig-pencil-high-level-picture}
} 
\end{figure}

Figure \ref{fig-pencil-high-level-picture} shows a high level
overview of how we envision \pencil to be used.
First, a program written in a domain
specific language is compiled into \pencil.  Domain specific
optimizations are applied during this translation, and the DSL compiler may
add domain specific information during this step through specific \pencil
language
constructs.  Second, the generated \pencil code is combined with
hand-written \pencil code that implements specific library
functions.  \pencil is used here as a standalone language.  The
combination of the two pieces of code is then optimized and
parallelized.  Finally, 
highly specialized low-level code is generated.
In Figure~\ref{fig-pencil-high-level-picture}, we
illustrate the case where the generated low-level code is OpenCL, 
allowing the compiled code to run across a range of OpenCL-compliant 
devices.  We hereafter assume that OpenCL is the target language for 
\pencil compilation, but in principle \pencil can target any suitable 
low-level representation.

The design of the extensions to C99 that are a part of \pencil went
through two steps.
First numerous DSLs (and benchmarks) were analyzed, and based on this
analysis,  a list of the properties that are expressed in these DSLs
was created.
This list was then filtered and only few properties were kept.
Deciding which properties were supposed to be expressed in \pencil
was guided by the following design choice:
all domain-specific optimizations should be performed at the DSL compiler
level, while the \pencil compiler should be responsible only for parallelization,
data locality optimization, loop nest transformations, and mapping to OpenCL.
This separation meant that, in \pencil, only the properties that are necessary
to enhance static-analysis and enable mapping to accelerator platforms are needed.
This choice has the advantage of keeping \pencil general-purpose,
semantically sequential and lightweight.

Section~\ref{pencil-c99-subset} presents the subset of C99 that
defines the core of \pencil, while Section~\ref{pencil-extension-short}
presents the extensions to C99 that are a part of \pencil.
The language syntax is defined in detaile in \autoref{pencil-ebnf} as an EBNF (Extended
Backus-Naur Form) grammar~\cite{wirth1996EBNF}.
\nomenclature{EBNF}{Extended Backus-Naur Form}

\subsection{\pencil Definition as a Subset of C99}
\label{pencil-c99-subset}

\subsubsection{Program Scope}
The following C99~\cite{c99} global definitions and declarations are allowed inside a
\pencil program:
\begin{itemize}
  \item Type definition;
  \item Function declaration and definition;
  \item Constant declaration: \pencil allows users to declare global constants
        as in C99, but it does not allow users to declare non-constant variables
        as global variables.
        This restriction enables \pencil compilers to
        assume that \pencil code does not have any side effect on global
        variables.
\end{itemize}

\subsubsection{Expressions (and Quasi-affine Expressions)}
\pencil supports a strict subset of C99 expressions:
\begin{itemize}
  \item Arithmetic, logical, bit and comparison operations.
  \item Array and member operators: \lstinline{[],.}
  \item Ternary conditional: \lstinline!cond?op1:op2!
  \item Size-of: \lstinline!sizeof(arg)!
  \item Scalar conversion: \lstinline!(type)arg!
  \item \pencil function calls.
  \item C function call (when a summary function is provided,
  details in Section~\ref{sec:summaries}).
\end{itemize}


\paragraph{Quasi-affine Expressions}
\label{sec:quasi-affine}

A quasi-affine expression is any expression over integer values and
integer variables involving only the operators \lstinline{+},
\lstinline{-} (both unary and binary), \lstinline{*}, \lstinline{/},
\lstinline
operators is required to be a (positive) integer literal, while at
least one of the arguments of the \lstinline{*} operator is
required to be piece-wise constant expression. An example of a
quasi-affine expression is: $a*i+b*j+c$, where $a$ and $b$ are
constants and $c$ is either a constant or a loop parameter.

\begin{lstlisting}[language=pencil]
for (int i = 1; i < n; i++) {
  A[10*i+20] = 0;       // The subscript of A[] is quasi-affine
  A[i*n] = 0;           // The subscript of A[] is not quasi-affine
  B[i*i] = 0;           // The subscript of B[] is not quasi-affine
  C[t[i]] = 0;          // The subscript of C[] is not quasi-affine
  D[foo(i)] = 0;        // The subscript of D[] is not quasi-affine
}
\end{lstlisting}

It is recommended to use quasi-affine expressions for array subscripts,
loop bounds and conditional expressions whenever this is possible.
In presence of non quasi-affine subscripts, it is highly recommended
to use the \lstinline!independent! or \lstinline!ivdep! directives 
(described in Section~\ref{sec:directives})
in order to enable parallelization and loop optimizations, or to hide these
accesses in functions annotated with summary functions
(Section~\ref{sec:summaries}).


\subsubsection{Types \label{penciltypes}}

Unlike C99, unions and bitfields are not supported in \pencil.
Only the following C99 types are allowed.

 \begin{itemize}
  \item Scalar types: scalar data types in \pencil are the same as in C99, with the
      addition of the optional \lstinline!half! float type
      (\pencil compilers are not required to support the \lstinline!half!
      float type).
  \item Structural types (same as in C99)
  \item Array types
  \begin{itemize}
    \item Arrays (including array function arguments) must be declared using the C99
      variable-length array syntax~\cite{c99}.
    \item Array function arguments
      must be declared using the \lstinline!static! keyword and
      the \lstinline!const! and \lstinline!restrict! type qualifiers
      described later
      (the macro \lstinline!pencil_array! can be used to
      abbreviate \lstinline!static! \lstinline!const! \lstinline!restrict!).
    \item Array accesses should not be linearized, as this tends to
      obfuscate affine subscript expressions and may reduce the quality of
      the data dependence analysis.
      Multidimensional C99 arrays (C99 variable-length arrays syntax)
      should be used instead.
    \item In order to have a precise dependence analysis, it is recommended
      to use quasi-affine array subscripts
      whenever this is possible.
  \end{itemize}
  \item Pointer types
  \begin{itemize}
    \item The declaration of pointers is allowed in \pencil.
      The main motivation is to provide \pencil users with the ability
      to use non-\pencil libraries in a \pencil code.
      Forbidding pointer declarations makes the use of non-\pencil libraries
      difficult since the header files of such libraries may contain pointer
      declarations.

    \item Pointer manipulation (including pointer arithmetic and reading or
      writing to a pointer) is not allowed.  There is one exception to this
      restriction, reading an array reference is allowed when the reference is
      passed in a function call (e.g., \lstinline!mat_add(C, A, B)!).
      
      The main motivation is to guarantee that the following property is
      preserved:
      throughout the life of a \pencil program, separate array references
      never alias and remain constant.
      Preserving this property is necessary to avoid the need for an
      advanced pointer
      analysis in \pencil compilers.
      Passing an array reference to a function is allowed in \pencil as it does not violate the previous
      property.  The property is not violated because function arguments in \pencil are required to be
      qualified with \lstinline!restrict! and \lstinline!const!:
      if two separate arrays are passed to a function and if they
      are qualified with the
      \lstinline!restrict! type qualifier then they are guaranteed
      not to alias withing that function.
      Moreover, the \lstinline!const! type qualifier guarantees
      that those array references remain constant within that function.

    \item Pointer dereferencing is not allowed.  The only exception to
      this is accessing arrays using array subscripts
      (e.g., \lstinline!A[i][j]!).

      Forbidding pointer arithmetic and forbidding pointer dereferencing
      (except dereferencing through array subscripts) makes the use of
      array subscripts to access an array element the unique way to do so
      which simplifies compiler analyses.
  \end{itemize}
    The restricted use of pointers is important for dependence analysis and
    for moving data between
    different address spaces of hardware accelerators, as it essentially
    eliminates aliasing problems.
 \end{itemize}

Scalar type conversion and type definition (through \lstinline!typedef!) in \pencil
both follow the same rules as in C99.

\subsubsection{Functions}
Function definition and declaration in \pencil follow the same
rules as in C99.
A function defined in \pencil can be called from \pencil
or from C99 code.
Function recursion is not allowed in \pencil since OpenCL does
not support recursion.
Function overloading is also not allowed as the C99 standard
does not allow it.


\subsubsection{Statements}
Unlike C99, \lstinline!goto! and \lstinline!switch! statements
are not allowed in \pencil.
Only the following C99 statements are allowed.

\paragraph{Assignment Statement}
\label{sec:assignments}
The following C99 assignment statements are supported in \pencil:
\begin{itemize}
  \item Basic assignment: \lstinline!=!
  \item Compound assignment: \lstinline{+=, -=, *=, /=, %=, |=, &=, ^=, <<=, >>=}
\end{itemize}

Unlike in C99, the \pencil assignment does not return any value (i.e.,
it is a statement, not an operator).
The motivation for this restriction is to simplify the development of
\pencil tool chains.
Example:
\begin{lstlisting}[language=pencil]
  int i = 2;          //Valid in C99 and in \pencil.
  int j = i += 2;     //Valid in C99, invalid in \pencil.
\end{lstlisting}

For convenience, \lstinline!i += 1! can be written as \lstinline!i++!
and \lstinline!i -= 1! can be written as \lstinline!i--!.

\paragraph{For Loop}

A \pencil for loop must have a single iterator, an invariant start
value, an invariant stop value and a literal constant increment (step).
Invariant in this context means that the value does not change within
the loop body.
The iteration variable must not be visible (defined) outside
the loop and cannot be modified
inside the loop body.

\begin{lstlisting}[language=pencil]
  for (type iter = init; iter [<|<=|>|>=] bound; iter[+|-]=step)
  {
    //Body
  }
\end{lstlisting}
By precisely specifying the loop format, we avoid the need for
a sophisticated induction variable analysis.
Such an analysis is not only complex to implement, but more
importantly results in compiler analyses succeeding or failing under
conditions unpredictable to the user.


\paragraph{While Loop}
The same as in C99.
\paragraph{If Statement}
The same as in C99.
\paragraph{Break and Continue Statements}
The same as in C99.
\paragraph{Return Statement}
The same as in C99 except that it can be used only at the end of a function
(i.e., as the last statement in a function body).
This allows \pencil compilers to assume that they work on SESE
(Single-Entry Single-Exit) regions of the control flow, which is easier.
\nomenclature{SESE}{Single-Entry Single-Exit}

\paragraph{Call Statement}
The same as in C99 except that
\begin{itemize}
 \item Calling a non-\pencil function from \pencil
   is allowed only if its summary function is provided
   (details about summary functions are presented in
    Section~\ref{sec:summaries}).
 \item Recursive function calls are not allowed (one reason
 is that recursive function calls are not supported in OpenCL).
\end{itemize}

\subsubsection{Identifiers, Declarations and Initializations}
The naming of identifiers, the scope of identifiers, identifier
declaration and initialization, in \pencil, all follow the same rules as
in C99.

\subsubsection{Preprocessor}
\pencil provides preprocessing directives equivalent to the C99
preprocessing directives.
A normal C99 preprocessor can be used to preprocess \pencil code.

\subsection{\pencil Extensions to C99}
\label{pencil-extension-short}

This section provides a list of \pencil extensions.  A detailed
description of these extensions is provided
in Section~\ref{sec:Annotations-and-directives}.

\subsubsection*{Directives}
\label{sec:for-directives}

\begin{itemize}
\item \lstinline!#pragma pencil independent [OPT[reduction(op: scal_1, $\ldots$, scal_n)]OPT]*!
\item \lstinline!#pragma pencil ivdep!
\item \lstinline!#pragma pencil region!
\end{itemize}

\subsubsection*{Function Attributes}

\begin{description}
  \item \lstinline!__attribute__((pencil_access()))!
  \item \lstinline!__attribute__((const))!
  \item \lstinline!__attribute__((pencil))!
\end{description}

\subsubsection*{Builtin Functions}

\begin{itemize}
\item \lstinline!__pencil_kill!;
\item \lstinline!__pencil_use!;
\item \lstinline!__pencil_def!;
\item \lstinline!__pencil_maybe!;
\item \lstinline!__pencil_assume!;
\item \lstinline!__pencil_assert!;
\item \pencil math, common and integer builtin functions.
\end{itemize}

\section{Detailed Description}
\label{sec:Annotations-and-directives}

\subsection{\texttt{static}, \texttt{const} and \texttt{restrict}}
\label{sec:array-type-qualifiers-section}

\subsubsection{\texttt{static} Keyword}
\label{sec:static}


The \lstinline!static! keyword is used when declaring an array argument of a \pencil function.
It
has the same semantics as the \lstinline!static! keyword in C99.
All array arguments of a \pencil function
must be declared using this keyword.
The use of this keyword is
important to implement array expansion or data transfers when
subscripts are not affine.%
\footnote{An array expansion maps an array to a larger array, typically
by adding extra dimensions.  The mapping may depend on the statement
instance and can be used to remove some memory reuse.}


\subsubsection{\texttt{const} Type Qualifier}
\label{sec:const}

The \lstinline!const! type qualifier is used when declaring an array argument of a \pencil function.
It
has the same semantics as the \lstinline!const! type qualifier in C99.
All array arguments of a \pencil function
must be declared using this type qualifier.
It is used to make sure that array arguments behave as closely as possible to
array variables, and to forbid that array arguments (the base
pointer, not individual elements) occur at the left-hand side of an
expression.  This rule eliminates the risk of inducing array aliasing
through the assignment of an arbitrary base address to an array
argument.

\subsubsection{\texttt{restrict} Type Qualifier}
\label{sec:restrict}

% A discussion about whether the restrict keyword should be kept in PENCIL
% or not.  The advantage of keeping restrict is to be compatible with the
% C programming style where safety is not the default, and safety (restrict)
% is indicated using a keyword.


The \lstinline!restrict! type qualifier is used when declaring an array argument of a \pencil function.
  It has the same syntax and semantics
  as in C99.
  All array arguments of a \pencil function
  must be declared using this type qualifier.
  The use of \lstinline!restrict! guarantees that the array
  arguments of the function may only alias if they are
  being used for read only.
  This allows \pencil compilers to
  perform more precise dependence analysis.
  
  Note that making the use of \lstinline!restrict! mandatory in \pencil requires
  upstream versioning of functions: if two arguments of a function are known
  to alias, then the two arguments must be transformed into one argument to
  avoid the aliasing problem.

\subsubsection*{Examples}

Here is a correct \pencil function declaration and call.

\begin{lstlisting}[language=pencil]
/* PENCIL code.
 * a and b are restricted arrays (analogous to restricted pointers in C99).
 */
void foo(int n, float a[static const restrict n]
                  float b[static const restrict n]);


void bar()
{
  int n = 42;
  float pa[n], dc[n];

  foo(n, pa, dc);
}

\end{lstlisting}

%\todo{Perhaps we should make an exception for
%   dynamic memory allocation.  In that case, the allocated address
%   should be held in a pointer-to-array type, not a
%   pointer-to-scalar/struct type in order to preserve size information.
%   An old revision (svn r552) of this document has an example, but the
%   details are still in the works.}

\lstinline!pencil_array! is a macro that abbreviates
  \lstinline!static const restrict!.

  
\subsection{Description of the Memory Accesses of Functions}
\label{sec:summaries}

% Note: Semantics (albert):
% - it is plain C code with some restrictions that the only non-scalar
%   accesses are USE/DEF with;
% - the semantics of USE/DEF is nop; as a corollary, a summary function has
%   no side effect, it is pure;
% - calls to other summaries are allowed.

  The effect of a function call on its array arguments is derived
  from an analysis of the called function.
  In some cases, the results of this analysis may be too inaccurate.
  In the extreme case, no code may be available for the function and the
  compiler can then only assume that every element of the passed arrays
  is accessed.
  In order to obtain more accurate information on memory accesses, the user
  may tell the compiler to derive the memory accesses not from the actual
  function body, but from some other function with the same signature.
  Such a function is called a \emph{summary function}.

  Summary functions are used to:
    \begin{itemize}
     \item Describe the memory accesses of library functions
     called from \pencil code --- as library functions cannot be analyzed at
     compile time.
     
     \item Describe the memory accesses of non-\pencil
     functions called from \pencil code --- as they are difficult
     to analyze.

     \item Describe the memory accesses of \pencil functions with complex
     memory access patterns.
     Although the compiler can perform memory access analysis automatically,
     it may perform a conservative analysis and may over-approximate the
     actual access patterns.  In this case, memory access information should
     be specified.
    \end{itemize}

  The use of summary functions in these cases enables more
  precise static analysis.
  To indicate the summary function of a function \lstinline!foo()!,
  one uses the attribute
  \lstinline!pencil_access(summary)!, where
  \lstinline!summary! is the name of the summary function
  that describes the memory accesses in \lstinline!foo()!.

  A summary function is not meant to be executed, and is instead only used
  for the analysis of memory footprints.
  It has the same arguments as the qualified function.
  Each and every array element accessed in a function should be accessed
  in its summary. Yet a summary is generally simpler than the function
  it summarizes: it only captures sets of accesses, not their ordering
  and number of occurrences.
  The semantics of memory access information are equivalent to the semantics
  of interprocedural function summaries.  A simple way to integrate
  it in an existing intraprocedural framework is to inline all
  summaries.

  The polymorphic builtin functions \lstinline!__pencil_use!
  and \lstinline!__pencil_def! must be used in summary
  functions to mark memory access information (and to protect them
  from aggressive, \pencil-agnostic upstream passes).
  The builtin function \lstinline!__pencil_use!, abbreviated with
  \lstinline!USE!, annotates read accesses, while \lstinline!__pencil_def!,
  abbreviated with \lstinline!DEF!, annotates (must-)write accesses.
  The polymorphic argument of these builtin functions may be a scalar,
  dereferenced pointer argument, or array element. It may also be a
  complete array when the dimension and size of the array are
  statically known:
  e.g., \lstinline!__pencil_use(A)! marks the use of the
  complete array A, alleviating the need to list every
  element.
  
  A summary function can contain calls to other functions, indicating
  that corresponding calls are present on the original function.  Example:

  \begin{lstlisting}[language=pencil]
    void foo(int N, int A[pencil_array N]);

    void bar_summary(int N, int A[pencil_array N])
    {
      foo(N, A);

      for (int i = 0; i < N; i ++)
      {
        USE(A[i]);
        DEF(A[i]);
      }
    }

    void bar(int N, int A[pencil_array N])
      __attribute__((pencil_access(bar_summary)))
      {
        foo(N, A);

        for (int i = 0; i < N; i ++)
        {
          int t = A[i];
          t = t + 1;
          A[i] = t;
        }
      }
  \end{lstlisting}
  
  To express may-write accesses, the boolean builtin
  \lstinline!__pencil_maybe! should be used to guard these accesses
  in an \lstinline!if (__pencil_maybe())!  conditional. We use the
  \lstinline!MAYBE! macro as an abbreviation. The
  builtin may be combined with more (affine) conditions to refine
  the static may-execute information. Example:

  \lstinline!if (__pencil_maybe()) __pencil_def(v))!

  One may also consider \lstinline!MAY_DEF(v)! as a short-cut that
  can replace the above example.

  The main criteria for choosing a given formalism instead of another
  formalism to express the memory accesses of a function are:
  \begin{itemize}
   \item the expressiveness of the formalism;
   \item the ease of expressing memory accesses in the formalism;
   \item the ability to use existing compilation frameworks to parse
         and analyze the formalism.
  \end{itemize}

  Summary functions were chosen over other formalisms (e.g., C++11 lambda
  functions~\cite{cpp11}, the use of directives, etc.) because summary
  functions are based on the C syntax and thus can be parsed and analyzed
  easily using existing C compiler APIs.
  Moreover, summary functions are expressive and allow a fine-grained
  specification of memory accesses.
  More importantly, summary functions are easy to understand unlike other
  formalisms such as lambda functions (not all C programmers are familiar with
  the concept of lambda functions).
  
  A summary function should be \pencil compliant so that \pencil tools can
  analyze it.
  Writing the summaries of library functions is the library developer's 
  responsibility.
  Such summary functions should be provided in the library's header
  files and are used directly by the DSL compilers or \pencil programmers.
  This is the most common case.
  In other less common cases, summary functions are either written by the
  \pencil programmer himself or generated automatically by the DSL compiler
  (only the sets of read and written elements for each function argument need to 
  be provided in this case).
  
  The attribute \lstinline!pencil_access! is
  abbreviated with the \lstinline!ACCESS! macro.

\paragraph{Example 1}
  \begin{lstlisting}[language=pencil]
  __attribute__((pencil_access(summary_fft32)))
  void fft32(int i, int j, int n,
              float in[pencil_attributes n][n][n]);

  int ABF(int n, float in[pencil_attributes n][n][n])
  {
    // ...
    for (int i = 0; i < n; i++)
    {
      // ...
      for (int j = 0; j < n; j++)
        fft32(i, j, n, in);
    }
    // ...
  }

  void summary_fft32(int i, int j, int n,
              float in[pencil_attributes n][n][n]);
  {
    for (int k = 0; k < 32; k++)
      __pencil_use(in[i][j][k]);
    for (int k = 0; k < 32; k++)
      __pencil_def(in[i][j][k]);
  }
  \end{lstlisting}
  
  This example shows a loop nest extracted from
  \emph{ABF} (Adaptive Beamformer)\nomenclature{ABF}{Adaptive Beamformer},
  a signal processing kernel used in radar
  systems.
  The code calls the function \lstinline!fft32! (Fast Fourier Transform)
  which only reads and modifies (in place) 32 elements of its input
  array \lstinline!in!, rather than modifying the whole input array.
  Such a function is not analyzed by the \pencil compiler as it is not a \pencil
  function.
  Without a summary function, the compiler assumes conservatively that the whole
  array passed to \lstinline!fft32! is accessed for read and for write.
  Such a conservative assumption prevents the parallelization of the code.
  The use of a summary function in this case indicates to the compiler that
  each iteration of the loop nest reads and writes 32 elements
  of the input array.
  This information allows the compiler to parallelize and optimize the loop
  nest.

  
\paragraph{Example 2}
  \begin{lstlisting}[language=pencil]
struct complex {
  int image;
  int real;
};

typedef struct complex Cplx;

void foo_summary(int n, int A[pencil_array n],
                       Cplx d[pencil_array n])
{
  for (int i = 0; i < n; i++)
    USE(A[i]);

  /* Note that i starts from 10. */
  for (int i = 10; i < n; i++)
    MAY_DEF(d[i].real);

  DEF(A[15]);
}

void foo(int n, int A[pencil_array n],
            Cplx d[pencil_array n])
            ACCESS(foo_summary(n,A,Cplx))
{
  int t;

  for (int i = 0; i < n; i++)
    printf("Value=%d", A[i]);

  for (int i = 0; i < n; i++)
  {
    if (A[i])
      printf("%d", i);

    t += A[i];
  }

  for (int i = 0; i < n; i++)
    if (A[i] && i>10)
      d[i].real = t;

  A[15] = 0;
}
  \end{lstlisting}

  In the above example, the summary function
  \lstinline!foo_summary!  indicates that the function
  \lstinline!foo! reads the values of \lstinline!A[i]!  for $i$
  from $0$ to $n-1$ and writes to \lstinline!A[15]!.
  \lstinline!foo! may write to \lstinline!d[i].real! for $i$
  from $10$ to $n-1$. 

\subsection{\texttt{const} Function Attribute}
\label{sec:const-attribute}

\nomenclature{GCC}{GNU Compiler Collection}
The \lstinline!const! attribute used in the GCC compiler is allowed in
\pencil for compatibility with existing \lstinline!const!-annotated
library functions (e.g., \lstinline!cos!, \lstinline!sin!, \lstinline!min!,
\lstinline!max!, etc.).
It indicates that the function does
not read any value except its arguments, and does not have any
effects except its return value~\cite{gccguide}.
The \lstinline!CONST! macro can be used to abbreviate
\lstinline!__attribute__((const))!.
  
\subsection{Pragma Directives\label{sec:directives}}

\pencil defines several directives inspired by OpenMP~\cite{openmp08} and OpenACC~\cite{openacc11}
and commonly found in advanced vectorizing compilers.
\nomenclature{OpenMP}{Open Multi-Processing}
\nomenclature{OpenACC}{Open Accelerators}

\subsubsection{\texttt{independent} Directive\label{sec:independent}}


The \lstinline!independent! directive is used to annotate loops.
It has the following form:
\begin{lstlisting}[language=pencil]
#pragma pencil independent [OPT[reduction(op: scal_1, $\ldots$, scal_n)]OPT]*
\end{lstlisting}


% It indicates that for a given execution of the marked loop, if two
% distinct iterations access the same memory element of a shared
% variable, then both accesses are read accesses A variable is
% considered shared if it is not declared inside the loop body.

The directive indicates that the desired result of the annotated
loop does not depend upon the execution order of data accesses
from different iterations.  The definition of data accesses
encompasses all memory accesses enclosed by the loop, either directly
or indirectly through calls to \pencil or non-\pencil functions. In
particular, data accesses from different iterations may be executed
simultaneously. The definition of desired result is algorithm- and
application-dependent.

The execution order of data accesses may be entirely defined by
the dependences of the \pencil program, in which case the semantics of
the pragma is portable. Alternatively,
some dependences may exist and nonetheless
ignored through the usage of the \lstinline!independent! directive,
in which case the correct execution order may have to be guaranteed
using specific synchronization constructs.
Reductions implemented as atomic regions in the generated code are one
typical example. Low-level atomics in C11~\cite{c11} or OpenCL 2.0 are another one
(e.g., to give semantics to benign races).
\nomenclature{C11}{C Standard 2011}

External non-\pencil functions called from the annotated loops
may employ target-dependent constructs to protect the atomicity of
their data access sequences, or to refine their parallel semantics
regarding relaxed memory ordering, sound implementation of benign
races, etc.
Such an approach is
currently necessary to allow benign races (when the same value is
written by multiple threads), to parallelize associative and commutative
operations, and more generally for any parallel algorithm tolerating
the unordered execution of intermediate steps.

The \lstinline!independent! directive has an effect only on the marked loop
and does not have an effect on any other loop in the loop nest.
It is typically used to indicate
that the annotated loop does not carry any dependence.  It allows the
compiler to ignore all loop carried dependences of the annotated loop,
including those that may be introduced by the compiler due to
conservative assumptions (for example, if the code contains non affine
write accesses).
Note that no implicit privatization of scalars and arrays is assumed when
the \lstinline!independent! directive is used.  Instead, scalars and
arrays that are privatizable should be declared as local variables
within the scope of the annotated loop.
This may sound overly restrictive, but it ensures the portability of
\pencil when targeting languages and architectures where races have
undefined semantics (C11, OpenCL 2.0).

\paragraph{Note}
The independent directive currently has informal semantics only. There
are plans for more formal versions, or for the complete deprecation of
the directive, in future revisions of \pencil.

\paragraph{Example 1}
  The following example shows a code fragment of a \pencil
  implementation of the breadth-first search algorithm.
  This algorithm computes the minimal distance from a given source node to
  each node of the input graph.
  The algorithm maintains a frontier and computes the next frontier by examining
  all unvisited nodes adjacent to the nodes of the current frontier.
  All nodes in a frontier have the same distance from the source node.

  The \lstinline|for| loop shown in the example can be
  parallelized since each node of the current frontier can be processed
  independently.
  This creates a possible race condition on the \lstinline|cost| and
  \lstinline|next_frontier| arrays.
  The race condition can be ignored, however, because each conflicting thread
  will write the same values to the arrays.
  By specifying the \lstinline|independent| pragma, the programmer guarantees
  that the race condition is benign, which enables a \pencil compiler to
  parallelize the loop.

  \begin{lstlisting}[language=pencil]
  /* Examine nodes adjacent to current frontier */
  #pragma pencil independent
  for (int i = 0; i < n_nodes; i++) {
    if (frontier[i] == 1) {
      frontier[i] = 0;
      /* For each adjacent edge j */
      for (int j = edge_idx[i];
                j < edge_idx[i] + edge_cnt[i]; j++) {
        int dst_node = dst_node_index[j];
        if (visited[dst_node] == 0) {
          /* benign race: threads write same values */
          cost[dst_node] = cost[i] + 1;
          next_frontier[dst_node] = 1;
        }
      }
    }
  }
  \end{lstlisting}

\paragraph{Example 2}

  The following example illustrates the use of the directive in a
  typical context where the code executed in the loop body executes
  target-dependent, non-\pencil code, e.g., code with atomic execution
  constraints that are currently not expressible in \pencil.
  \begin{lstlisting}[language=pencil]
void inc_summary(float A[256], int c)
{
  if (MAYBE)
  {
    USE(A);
    DEF(A);
  }
}

void inc(float A[pencil_array 256], int c) ACCESS(inc_summary);

void foo(int N, float A[pencil_array 256], int t[pencil_array N])
{
  #pragma pencil independent
  for (int i = 0; i < N; i++)
    inc(A, t[i]);
}  
  \end{lstlisting}

  The following non-\pencil C code implements \lstinline!inc!,
  and is provided in a different file and compiled separately.
  \begin{lstlisting}[language=pencil]
void inc(float A[256], int c)
{
  atomic_inc(&A[c]);
}
  \end{lstlisting}

  In this case, the user needs to provide an OpenCL implementation for
  \lstinline!inc! (an implementation that performs atomic increments).

\paragraph{Example 3}
  In the following example, the writes to \lstinline!t! induce loop carried
  dependences and thus the use of \lstinline!independent! is incorrect because
  there are at least two iterations that may write different values to
  \lstinline!t!.

  \begin{lstlisting}[language=pencil]
int t;
#pragma pencil independent
for (int i = 0; i < N; i++) {
  t = foo(i);
  A[B[i]] = t;
}
  \end{lstlisting}  
  
  To be able to use \lstinline!independent! in the previous example, the
  scalar \lstinline!t! has to be declared in the loop body. This way, each
  iteration will have its own copy of \lstinline!t! and thus the writes to
  \lstinline!t! by different iterations will not induce any loop carried
  dependence.  This is possible because \lstinline!t! is privatizable (i.e.,
  can be made private).
  The following example is correct (assuming that the elements of
  \lstinline!B! are all different).

  \begin{lstlisting}[language=pencil]
#pragma pencil independent
for (int i = 0; i < N; i++) {
  int t = foo(i);
  A[B[i]] = t;
}
  \end{lstlisting}
   
  In general, all the variables declared inside the loop body (whether the loop
  is marked as independent or not) can be assumed to be free of loop-carried
  dependences (i.e., these variables do not induce any loop carried dependence).
  
  \paragraph{\texttt{reduction} Clause}
  \label{reduction-clause}

  Adding the \lstinline!reduction! clause to the \lstinline!independent!
  directive restricts the execution order of data accesses with respect
  to \lstinline!independent! alone: considering 
  the execution order of data accesses to the reduction variables
  (\lstinline!scal_1, $\ldots$, scal_n!), the compiler must preserve
  the atomicity of side effects on these variables within a given
  loop iteration.
  This in turns widens the applicability of the directive, compared to
  \lstinline!independent! alone.
  The reduction operator (\lstinline!op!) itself is only useful to
  indicate how partial results on a reduction variable resulting from
  any interleaving should be combined.

  Aside from extending the applicability of the independent directive, one
  motivation for introducing the \lstinline!reduction!
  clause in \pencil is to eliminate the need for having a sophisticated
  analysis to detect reductions in \pencil compilers.

  In order to simplify code generation for \pencil compilers, only scalar
  variables can be used as reduction variables.
  % Additional reason: we also prefer to keep \pencil in sync with PPCG.
  % Support for scalar variables is relatively easier to add in PPCG than
  % support for array elements.
  % Thus we prefer not to claim that we can support reductions on array
  % elements to avoid making the gap between the \pencil spec and PPCG
  % smaller.  In practice, reductions on array elements are rare.
  Multiple reduction clauses can be used for different reduction operators.
  The syntax of this clause is equivalent to the syntax of the reduction clause
  defined in OpenMP.  As in OpenMP, the reduction variables should not
  be used anywhere outside the reduction statement.  Example:
  \begin{lstlisting}[language=pencil]
#pragma pencil independent reduction(+: result)
for (int i = 0; i < n; i++) {
  B[T[i]] = foo(i);
  result += A[i];
}
  \end{lstlisting}
  In the above example, the compiler will ignore all the loop carried
  dependences of the loop except the loop carried dependences induced by
  the reduction variable \lstinline!result! in the second statement.

The following reduction operators are supported:
\lstinline!+!, \lstinline!-!, \lstinline!*!, \lstinline!min!, \lstinline!max!.

%--------------------------------------------------------------------------------
\subsubsection{\texttt{ivdep} Directive\label{sec:ivdep}}

The \lstinline!ivdep! directive is used to annotate innermost loops that are candidates
  for vectorization.  If applied on a loop nest, it is effective only on
  the innermost loops of the nest.

  It has the following form:
  \begin{lstlisting}
  #pragma pencil ivdep
  \end{lstlisting}
  
  It allows the
  compiler to ignore all the loop-carried dependences in the loop marked with
  the directive from one statement to a textually earlier one (Cray semantics for ivdep~\cite{cray}).
  This is generally sufficient to enable the vectorization
  of loops when the compiler is taking conservative assumptions.

  The \lstinline!independent!  directive is stronger than the
  \lstinline!ivdep! directive, as the latter only guarantees the
  correctness of a lock-step parallel execution (i.e., an implicit
  synchronization barrier in between every pair of statements in the loop
  body).
  
  \paragraph{Example 1}

  \begin{lstlisting}[language=pencil]
  #pragma pencil ivdep
  for (int i = 0; i < m; i++)
  {
    float t = a[i + k] * c;
    a[i] = t;
  }
  \end{lstlisting}
  In this example, vectorization would be invalid if $k < 0$.  The
  \lstinline!ivdep! directive allows the compiler to ignore the assumed
  loop carried dependences that may exist if $k < 0$.\footnote{Ignore
  possible out-of-bound errors in this example.}
  
  \paragraph{Example 2}
  
  \begin{lstlisting}[language=pencil]
  #pragma pencil ivdep
  for (int i = 0; i < m; i++)
  {
     float t = a[b[i]] + 3;
     a[b[i]] = t;
  }
  \end{lstlisting}
  In this example, the compiler will ignore textually backward
  loop-carried dependences for the store into \lstinline!a[]!.


\subsection{\label{sec:builtins}Builtin Functions}

In this section, \lstinline!T! represents any valid \pencil type.

\subsubsection{\texttt{\_\_pencil\_use}, \texttt{\_\_pencil\_def} and \texttt{\_\_pencil\_maybe} Builtins}

  These builtins should be used in summary functions
  (a detailed description is provided in Section~\ref{sec:summaries}).
  They have the following prototypes:
  
  \lstinline!void __pencil_use(T location);!

  \lstinline!void __pencil_def(T location);!

  \lstinline!int __pencil_maybe();!

\subsubsection{\label{sec:kill}\texttt{\_\_pencil\_kill} Builtin}
  
The \lstinline!__pencil_kill! builtin function has the following prototype:

  \lstinline!void __pencil_kill(T location);!

  It allows the user to refine dataflow information
  within and across any control flow region in the program.
  It is a polymorphic function that signifies that its argument
  (a variable or an array element) is dead at the program point
  where \lstinline!__pencil_kill! is inserted, meaning that no data
  flows from any statement instance executed before the kill to any
  statement instance executed after.

  This information is used in two ways.
  \begin{itemize}
   \item In eliminating dataflow dependences within the control
         flow region.
   \item To determine which array elements may have their contents
         preserved by the region.
         In particular, when the region is mapped to a device kernel, then
         data that may be written inside the region and is possibly needed
         afterwards has to be copied back from the device to the host.
         The region of the array that is copied back to the host may
         however be larger than the set of elements that are known
         to be written by the region, either because some elements
         may only be written under certain circumstances or because
         the region that is copied back is an over-approximation.
         In such cases, the region first needs to be copied into the device
         to ensure that the elements within the array region that are not
         actually written by the code region preserve their original values.
         This latter step is not needed if these values were not preserved
         by the original code region.  Such information can be passed to
         the compiler using \lstinline!__pencil_kill!.
  \end{itemize}

  \lstinline!__pencil_kill! is abbreviated with the \lstinline!KILL! macro.

\paragraph{Example}
  In the following code, the elements of \lstinline!A!
  may be written inside the loop.  This means that if the loop
  is mapped to a device kernel, then this array needs to be copied
  out from the device to the host.  Since not all elements may be
  written by the loop, the array would in principle also need to
  be copied in first.  The \lstinline!__pencil_kill(A)! statement
  is used to indicate that the data is not expected to be preserved
  by the region and that therefore this copy-in can be omitted.

\begin{lstlisting}[language=pencil]
__pencil_kill(A);
for (int i = 0; i < n; i++) {
  if (B[i] > 0)
    A[i] = B[i];
}
\end{lstlisting}

\subsubsection{\texttt{\_\_pencil\_assume} Builtin}


The \lstinline!__pencil_assume! builtin function has the following prototype:

  \lstinline!void __pencil_assume(int expression);!

  It indicates that its argument (which is a logical expression) is true at
  the program point where \lstinline!__pencil_assume! is inserted.
  \lstinline!__pencil_assume(exp)! does not instruct the compiler to check at
  run time whether \lstinline!exp! is actually true or not.
  One may use \lstinline!__pencil_assert(exp)! for that purpose instead.
  In the context of DSL compilation, an assume statement allows a
  DSL-to-\pencil compiler to communicate high level facts in the generated
  code.

  \lstinline!__pencil_assume! is abbreviated with the \lstinline!ASSUME! macro.
  
\paragraph{Example 1}
The code in Figure~\ref{fig:2D-convolution} (a general 2D convolution)  is a
good example where the assume builtin can be used.  It is an image
processing kernel that calculates the weighted sum of the area around each
pixel using a kernel matrix for weights.
In this code, it is sufficient to consider that the size of the array
\lstinline!kern_mat! does not exceed $15\times15$ (Such information is
used implicitly in the OpenCV OpenCL image processing library for example).
In fact, most convolutions do not exceed a kernel matrix size of $5\times5$

While this information is well known for an image processing expert, the
compiler does not have this knowledge and must assume that the kernel matrix
can be arbitrarily large.  When compiling for a GPU
\nomenclature{GPU}{Graphics Processing Unit} target, the compiler must
thus allocate the kernel matrix in GPU global memory, rather than fast shared
memory, or must generate multiple variants of the kernel --- one to handle large
kernel matrix sizes and another optimized for smaller kernel matrix
sizes --- selecting between variants at runtime.
The use of \lstinline!__pencil_assume! in this case tells the compiler
about the limits on the size of the array, allowing it to store
the whole array in shared memory.

\begin{figure}[t]
\begin{lstlisting}[stepnumber=1,numbers=left,numberstyle={\tiny\tt},numbersep=5pt,escapechar=@,language=pencil]
#define clampi(val, min, max) \
  (val < min) ? (min) : (val > max ) ? (max):(val)

__pencil_assume(ker_mat_rows <= 15);@\label{fig:2D-convolution:assume1}@
__pencil_assume(ker_mat_cols <= 15);@\label{fig:2D-convolution:assume2}@

for (int i = 0; i < rows; i++)
  for (int j = 0; j < cols; j++) {
    float prod = 0.;@\label{fig:2D-convolution:declaration}@
    for (int e = 0; e < ker_mat_rows; e++)
      for (int r = 0; r < ker_mat_cols; r++) {
        row = clampi(i+e-ker_mat_rows/2, 0, rows-1);
        col = clampi(j+r-ker_mat_cols/2, 0, cols-1);
        prod += src[row][col] * kern_mat[e][r];
      }
    conv[i][j] = prod;
  }
\end{lstlisting}
\caption{General 2D convolution}
\label{fig:2D-convolution}
\vskip-0.5cm
\end{figure}
  
\paragraph{Example 2}
  \begin{lstlisting}[language=pencil]
void foo(int n, int m, int S, int D[pencil_array S])
{
        __pencil_assume(m > n);
        for (int i = 0; i < n; i++) {
                D[i] = D[i+m];
        }
}
  \end{lstlisting}
The loop above cannot be parallelized since it might have loop
carried dependences (if the step \lstinline!m! is less than the
number of iterations \lstinline!n!).
The \lstinline!__pencil_assume! builtin can be used to inform the compiler that
\lstinline!m! is greater than \lstinline!n!, which allows the compiler
to parallelize the loop.
An alternative solution in this particular case is to explicitly
mark the loop as independent.

  \begin{lstlisting}[language=pencil]
void foo(int n, int m, int S, int D[pencil_array S])
{
        #pragma pencil independent
        for (int i = 0; i < n; i++) {
                D[i] = D[i+m];
        }
}
  \end{lstlisting}


\subsubsection{\texttt{\_\_pencil\_assert} Builtin}

The \lstinline!__pencil_assert! builtin function has the following prototype:

\lstinline!void __pencil_assert(int expression);!

  It instructs the compiler to insert a run-time check of whether its argument
  (which is a boolean expression) is true at the program point where
  \lstinline!__pencil_assert! is inserted.
  \lstinline!__pencil_assert(exp)! does not automatically imply
  \lstinline!__pencil_assume(exp)!: static analyses in a \pencil compiler may
  ignore the assert builtin while relying on
  the assume builtin for enhanced analyses accuracy or speed.  If the
  target architecture does not support the assert builtin (OpenCL for
  example), the \pencil compiler should emit a warning message.

\subsubsection{\pencil Math, Common and Integer Builtin Functions}

\pencil supports a set of the OpenCL builtin functions:
\begin{itemize}
  \item all OpenCL integer functions (\lstinline!abs!, \lstinline!clz!,
        \lstinline!popcount!,  \dots);
  \item all OpenCL common functions (\lstinline!min!, \lstinline!max!,
        \lstinline!clamp!, \lstinline!sign!, \dots);
  \item all OpenCL math functions (\lstinline!sin!, \lstinline!exp!,
        \lstinline!cos!, \lstinline!log!, \dots).
\end{itemize}

The full list of OpenCL integer, common and math builtin functions is
available in \cite{opencl-1.2}.

As \pencil does not support function overloading (to be consistent
with C99), every OpenCL builtin integer, common or math function
has multiple equivalent functions in \pencil with prefixes and
suffixes for the different argument types.
For example, the OpenCL \emph{max} function has the
following equivalent \pencil functions
\begin{itemize}
 \item \emph{max}: used for int arguments;
 \item \emph{smax}: used for short arguments;
 \item \emph{bmax}: used for char arguments;
 \item \emph{fmax}: used for double arguments;
 \item \emph{lmax}: used for long arguments;
 \item \emph{fmaxf}: used for float arguments;
 \item unsigned versions have a \emph{u} prefix
 (\emph{ubmax}, \emph{usmax}, \emph{umax}, \emph{ulmax}).
\end{itemize}

\pencil includes scalar builtin functions to help vectorize specific
idioms. In particular, saturated and clamped arithmetic, absolute
value, min, max, etc.  Developers and DSL front-ends can use these
functions and rely on the polyhedral tools to generate a vectorized
version of these functions.

Floating point operations are considered associative by default in \pencil.
\pencil builtin functions do not have side-effects (on \texttt{errno}%
\footnote{\texttt{errno} is an integer variable set by system calls and
some library functions in case of an error to indicate what went wrong.})
and floating point operations do not trigger exceptions (note that these
assumptions match the effects of the \texttt{-fno-math-errno}
\texttt{-fno-signaling-nans} GCC flags~\cite{gccguide}).

\section{Practical \pencil Programming}

\subsection{Header Files}
\pencil code must always include
the \texttt{pencil.h} header file
which defines all the \pencil builtin functions and macros.%
\footnote{\texttt{pencil.h} is available from
\url{https://github.com/carpproject/}}
Header files that contain non-\pencil code (including many standard C header
files) are not supported in \pencil.


\subsection{Compiling \pencil with C99 Compilers}

To map \pencil code to OpenCL and to take advantage of all the features
of the \pencil language, a \pencil compiler needs to be used.
But it is still possible to compile \pencil code with a standard C99
compiler (gcc, icc, \dots) as if it were C99 code (useful mainly for
debugging).
\nomenclature{icc}{Intel C++ Compiler}
In that case all the \pencil extensions will be ignored by the C99
compiler.  Such a behavior is totally safe.

Note that \pencil compilers define the \lstinline!__PENCIL__! macro
for convenience.


\subsection{Embedding \pencil Code in C}
\label{sec:pencil-as-c-ext}

For convenience, it is possible to embed \pencil code in C code.
Embedded \pencil code obeys the same rules as standalone \pencil code.

\begin{itemize}
  \item \pencil functions can be embedded into a C program.
    Such functions must be annotated with \lstinline!__attribute__((pencil))!.
    \begin{lstlisting}[language=pencil]
      //C function.
      int foo(int *a)
      {
      }

      //PENCIL function.
      int foo_pencil(int a[pencil_array 10])
         __attribute__((pencil))
         {
         }
       \end{lstlisting}

  \item Blocks of \pencil code can also be embedded into a C function.
    Such blocks must be marked with the \lstinline!#pragma pencil region!:
    \begin{lstlisting}[language=pencil]
      //C function.
      int foo(int *a)
      {
        int S[20];
        #pragma pencil region
        {
          for (int i = 0; i < 20; i++)
          {
            S[i] = 0;
          }
        }
      }
    \end{lstlisting}
\end{itemize}

\subsection{Passing a Scalar by Reference to a Function}

A scalar argument that may be modified by a function (an output argument)
should be declared as a one-element array.

\begin{lstlisting}[language=pencil]
/* Valid in PENCIL. */
void set_zero(int a[pencil_array 1])
{
  a[0] = 0;
}

/* Invalid in PENCIL. */
void set_zero(int *a)
{
  *a = 0;
}
\end{lstlisting}  

\subsection{Array Memory Allocation in \pencil}

It is possible to allocate a local array dynamically in \pencil code
by declaring the array using the C99 VLA syntax (C99 Variable Length
Array syntax)~\cite{c99}. \nomenclature{VLA}{Variable Length Array}
An array is said to be local for a given \pencil region, if all the
elements of the array are defined in that \pencil region and are not
used outside that region.
All non-local arrays used in a given \pencil region must be allocated
outside that \pencil region.

\begin{lstlisting}[language=pencil]
/* C code.  fact[] is allocated here.  */
int *foo(int N)
{
    int *fact = malloc(sizeof(int) * N);
    set_array(N, fact, -1);
    return fact;
}

/* PENCIL code.  */
void set_array(int N, int array[pencil_array N],
                    int val)
{
    for (int i = 0; i < N; i++)
        array[i] = val;
}

\end{lstlisting}


\subsection{Providing the OpenCL Implementation of a \pencil Function}
Although for many practical examples \pencil enables unassisted
generation of sufficiently optimized OpenCL code,
there will inevitably be cases where a developer wishes
to hand optimize a specific piece of functionality, or exploit
non-standard, target-specific features of a particular platform (e.g.,
through the use of inline PTX assembly in OpenCL when targeting Nvidia
GPUs).
To make \pencil more flexible for expert developers and
domain-specific optimizers, the language allows the call of
target-specific functions within the generated code.
Such functions are provided by the user to the \pencil compiler
in a separate source file and are then included by the compiler
into the automatically generated OpenCL code.


\subsection{Examples of \pencil Code}

\subsubsection{BFS Example}

\nomenclature{BFS}{Breadth-First Search}

The following example is a parallel breadth-first search implementation
in \pencil.  The program takes as input a graph and a start node and
performs a breadth-first search.  The order in which the nodes are
visited (BFS order) is stored in the array \lstinline!cost!.  The algorithm has
two steps which repeat one after the other until all nodes have been explored.

In the first step, each node \emph{n} which is on the frontier (array \lstinline!front!
in the code) inspects its neighbors.  The neighbor of \emph{n}, if not visited previously,
is put on \lstinline!updating_front!, which is the set of nodes that will be on
the next front.  The cost of the neighbor is the cost of the node \emph{n} plus one.
Each node can perform this operation independently.  It is possible
that two nodes, both on the frontier,
have an edge to the same neighbor \emph{m}.  In that
case, both update the cost of this neighbor \emph{m} (Line \ref{l:inc} in the example).
This is a race (hazard) but it is a benign one since both will attempt to
write the same value.

In the second step, the nodes which have been put on \lstinline!updating_front!
are moved to the frontier (array \lstinline!front!).  Each node can perform
this step independently from the others.

One must note that the \pencil code below is much easier to read and understand
than the equivalent OpenCL code.


\begin{lstlisting}[language=pencil,escapechar=@, numbers=left,numberstyle={\tiny\tt},numbersep=5pt]

void parallel_version(int no_of_nodes,
  int edge_start_no[pencil_array no_of_nodes],
  int edge_count[pencil_array no_of_nodes],
  int no_of_edges,
  int dst_node_index[pencil_array no_of_edges],
  int cost[pencil_array no_of_edges],
  char front[pencil_array no_of_nodes],
  char updating_front[pencil_array no_of_nodes],
  char visited[pencil_array no_of_nodes],
  int src_index, int quiet)
{
  unsigned int carry_on = 1;

  for (int i = 0; i < no_of_nodes; i++)
  {
    front[i]= 0;
    updating_front[i] = 0;
    visited[i] = 0;
    cost[i] = -1;
  }

  front[src_index] = 1;
  visited[src_index] = 1;
  cost[src_index] = 0;

  #pragma pencil region
  {
    while (carry_on == 1)
    {
      carry_on = 0;

      #pragma pencil independent
      for (int i = 0; i < no_of_nodes; i++) {
        if(front[i] == 1) {
          front[i] = 0; 
          for (int j = edge_start_no[i]; 
              j < (edge_start_no[i] + edge_count[i]); j++) {
            int dst_node = dst_node_index[j];
            if (visited[dst_node] == 0) {
              cost[dst_node] = cost[i] + 1;@\label{l:inc}@
              updating_front[dst_node] = 1;
            }
          }
        }
      }

      for (int i = 0; i < no_of_nodes; i++) {
        if (updating_front[i] == 1) {
          front[i] = 1;
          visited[i] = 1;
          carry_on = 1;
          updating_front[i] = 0;
        }
      }  
    } //while.
  } //#pragma pencil region
}
\end{lstlisting}

\subsubsection{Image Resizing Example}

The following example implements an image resizing kernel,
a common image processing kernel.

\begin{lstlisting}[language=pencil,escapechar=@, numbers=left,numberstyle={\tiny\tt},numbersep=5pt]
#include <pencil.h>

#define bilinear(A00, A01, A11, A10, r, c) \
  ((1-c) * ((1-r) * A00 + r*A10) + c*((1-r)*A01 + r*A11))

static void resize(const int rows,
  const int cols,
  const int step,
  const unsigned char original[pencil_array rows][step],
  const int r_rows,
  const int r_cols,
  const int r_step,
  unsigned char resampled[pencil_array r_rows][r_step])
{
    __pencil_assume(rows  >  0);
    __pencil_assume(cols  >  0);
    __pencil_assume(step  >= cols);
    __pencil_assume(r_rows >  0);
    __pencil_assume(r_cols >  0);
    __pencil_assume(r_step >= r_cols);

    __pencil_kill(resampled);

    int o_h = rows;
    int o_w = cols;
    int n_h = r_rows;
    int n_w = r_cols;

    for (int n_r = 0; n_r < r_rows; n_r++)
    {
      for (int n_c = 0; n_c < r_cols; n_c++)
      {
        float o_r = (n_r + 0.5) * (o_h) / (n_h) - 0.5;
        float o_c = (n_c + 0.5) * (o_w) / (n_w) - 0.5;

        float r = o_r - floor(o_r);
        float c = o_c - floor(o_c);

        int coord_00_r = clamp((int) floor(o_r), 0, o_h - 1);
        int coord_00_c = clamp((int) floor(o_c), 0, o_w - 1);

        int coord_01_r = coord_00_r;
        int coord_01_c = clamp(coord_00_c + 1, 0, o_w - 1);

        int coord_10_r = clamp(coord_00_r + 1, 0, o_h - 1);
        int coord_10_c = coord_00_c;

        int coord_11_r = clamp(coord_00_r + 1, 0, o_h - 1);
        int coord_11_c = clamp(coord_00_c + 1, 0, o_w - 1);

        unsigned char A00 = original[coord_00_r][coord_00_c];
        unsigned char A10 = original[coord_10_r][coord_10_c];
        unsigned char A01 = original[coord_01_r][coord_01_c];
        unsigned char A11 = original[coord_11_r][coord_11_c];

        resampled[n_r][n_c] = bilinear(A00, A01, A11, A10, r, c);
      }
    }
    __pencil_kill(original);
}
\end{lstlisting}


\subsubsection{Gaussian Filter Example}

The following example implements a gaussian filter,
a common image processing kernel.

\begin{lstlisting}[language=pencil,escapechar=@, numbers=left,numberstyle={\tiny\tt},numbersep=5pt]
#include <pencil.h>

static void gaussian(const int rows,
  const int cols,
  const int step,
  const float src[pencil_array rows][step],
  const int kernelX_rows,
  const int kernelX_cols,
  const int kernelX_step,
  const float kernelX[pencil_array kernelX_rows][kernelX_step],
  const int kernelY_rows,
  const int kernelY_cols,
  const int kernelY_step,
  const float kernelY[pencil_array kernelY_rows][kernelY_step],
  float conv[pencil_array rows][step])
{
    __pencil_assume(rows         >  0);
    __pencil_assume(cols         >  0);
    __pencil_assume(step         >= cols);
    __pencil_assume(kernelX_rows >  0);
    __pencil_assume(kernelX_cols >  0);
    __pencil_assume(kernelX_step >= kernelX_cols);
    __pencil_assume(kernelY_rows >  0);
    __pencil_assume(kernelY_cols >  0);
    __pencil_assume(kernelY_step >= kernelY_cols);
    __pencil_assume(kernelX_rows <= 2);
    __pencil_assume(kernelX_cols <= 128);
    __pencil_assume(kernelY_rows <= 128);
    __pencil_assume(kernelY_cols <= 2);

    __pencil_kill(conv);

    float temp[rows][step];

    for (int q = 0; q < rows; q++)
    {
      for (int w = 0; w < cols; w++)
      {
        float prod = 0.;

        for (int e = 0; e < kernelX_rows; e++)
        {
          for (int r = 0; r < kernelX_cols; r++)
          {
            int row = clamp(q + e - kernelX_rows / 2, 0, rows-1);
            int col = clamp(w + r - kernelX_cols / 2, 0, cols-1);
            prod += src[row][col] * kernelX[e][r];
          }
        }
        temp[q][w] = prod;
      }
    }
    
    for (int q = 0; q < rows; q++)
    {
      for (int w = 0; w < cols; w++)
      {
        float prod = 0.;

        for (int e = 0; e < kernelY_rows; e++)
        {
          for (int r = 0; r < kernelY_cols; r++)
          {
            int row = clamp(q + e - kernelY_rows / 2, 0, rows-1);
            int col = clamp(w + r - kernelY_cols / 2, 0, cols-1);

            prod += temp[row][col] * kernelY[e][r];
          }
        }
        conv[q][w] = prod;
      }
    }

    __pencil_kill(kernelY);
    __pencil_kill(kernelX);
    __pencil_kill(src);
}
\end{lstlisting}


\subsection{Examples of non-\pencil Code}

\subsubsection{Recursive Data Structures and Recursive Function Calls}

The following code is not valid \pencil code.  The problems in this
code are the following.
\begin{itemize}
  \item \pencil does not allow recursive function calls;
  \item pointer dereferencing is not allowed in \pencil (except
        in a few cases cited in Section~\ref{penciltypes});
  \item in \pencil, the return statement should be used only at
        the end of the function.
\end{itemize}

\begin{lstlisting}[language=pencil,escapechar=@, numbers=left,numberstyle={\tiny\tt},numbersep=5pt]
 struct node
 {
   int value;
   struct node* left;
   struct node* right;
 };

 struct node* find(struct node* node, int value)
 {
   if (!node)
     return NULL;
   if (node->value == value)
     return node;
   if(value > node->value)
     return find(node->left);
   else
     return find(node->right);
 }
\end{lstlisting}

%TODO: add citations when necessary.

